\documentclass[czech,helvetica,openbib]{europecv}
\usepackage[T1]{fontenc}
\usepackage{graphicx}
\usepackage[a4paper,top=1.27cm,left=1cm,right=1cm,bottom=2cm]{geometry}
\usepackage[czech]{babel}
\usepackage{bibentry}
\usepackage{url}

\renewcommand{\ttdefault}{phv} % Uses Helvetica instead of fixed width font

\ecvname{Příjmení, Jméno}
\ecvfootername{Jméno Příjmení}
\ecvaddress{Ulice, číslo domu, město, PSČ, stát}
\ecvtelephone[(Pokud nepotřebujete, odstraňte)]{(Pokud nepotřebujete, odstraňte)}
\ecvfax{(Pokud nepotřebujete, odstraňte)}
\ecvemail{\url{email@address.cz} (Pokud nepotřebujete, odstraňte)}
\ecvnationality{(Pokud nepotřebujete, odstraňte)}
\ecvdateofbirth{(Pokud nepotřebujete, odstraňte)}
\ecvgender{(Pokud nepotřebujete, odstraňte)}
%\ecvpicture[width=2cm]{mypicture}
\ecvfootnote{Více informací najdete na \url{http://europass.cedefop.eu.int}\\
\textcopyright~European Communities, 2003.}

\begin{document}
\selectlanguage{czech}


\begin{europecv}
\ecvpersonalinfo[5pt]
\ecvitem{\large\textbf{Místo, o~které se~ucházíte}}{\large\textbf{(Pokud nepotřebujete, odstraňte)}}

\ecvsection{Pracovní zkušenosti}
\ecvitem{Datumy}{Zadejte oddělené položky pro každou relevantní pozici, začněte od poslední. (Pokud nepotřebujete, odstraňte).}
\ecvitem{Povolání nebo pozice}{\ldots}
\ecvitem{Hlavní aktivity a zodpovědnosti}{\ldots}
\ecvitem{Název a adresa zaměstnavatele}{\ldots}
\ecvitem{Odvětví}{\ldots}

\ecvsection{Vzdělání a certifikáty}

\ecvitem{Datumy}{Zadejte oddělené položky pro každý relevantní kurz, které jste dokončili, začněte od posledního. (Pokud nepotřebujete, odstraňte).}
\ecvitem{Název získaného vzdělání}{\ldots}
\ecvitem{Hlavní předměty/Získané zkušenosti}{\ldots}
\ecvitem{Název a typ organizace poskytující vzdělání}{\ldots}
\ecvitem{Úroveň podle vnitrstátní nebo mezinárodní klasifikace}{\ldots}

\ecvsection{Osobní dovednosti}

\ecvmothertongue[5pt]{Mateřský jazyk}
\ecvitem{\large Další jazyky}{}
\ecvlanguageheader{(*)}
\ecvlanguage{Jazyk}{}{}{}{}{}
\ecvlanguage{Jazyk}{}{}{}{}{}
\ecvlanguagefooter[10pt]{(*)}

\ecvitem[10pt]{\large Sociální dovednosti}{Nahraďte tento text popisem těchto zkušeností a uveďte, kde jste je získali (pokud nepotřebujete, odstraňte).}
\ecvitem[10pt]{\large Organizační dovednosti}{Nahraďte tento text popisem těchto zkušeností a uveďte, kde jste je získali (pokud nepotřebujete, odstraňte).}
\ecvitem[10pt]{\large Technické dovednosti}{Nahraďte tento text popisem těchto zkušeností a uveďte, kde jste je získali (pokud nepotřebujete, odstraňte).}
\ecvitem[10pt]{\large Počítačové dovednosti}{Nahraďte tento text popisem těchto zkušeností a uveďte, kde jste je získali (pokud nepotřebujete, odstraňte).}
\ecvitem[10pt]{\large Umělecké dovednosti}{Nahraďte tento text popisem těchto zkušeností a uveďte, kde jste je získali (pokud nepotřebujete, odstraňte).}
\ecvitem[10pt]{\large Ostatní dovednosti}{Nahraďte tento text popisem těchto zkušeností a uveďte, kde jste je získali (pokud nepotřebujete, odstraňte).}
\ecvitem{\large Řidičský průkaz}{Uveďte, zda máte řidičský průkaz a pro jaké kategorie. (Pokud nepotřebujete, odstraňte).}

\ecvsection{Ostatní informace}
\ecvitem[10pt]{}{Uveďte jakékoli další informace, které mohou být relevantní, např. kontaktní osoby, reference, atd. (Pokud nepotřebujete, odstraňte).}
\bibliographystyle{plain}
\nobibliography{publications}
\ecvitem{}{\textbf{Publikace}}
\ecvitem{}{\bibentry{pub1}}
\ecvitem[10pt]{}{\bibentry{pub2}}
\ecvitem{}{\textbf{Osobní zájmy}}
\ecvitem{}{\ldots}

\ecvsection{Přílohy}
\ecvitem{}{Uveďte seznam příloh k Vašemu CV}
\end{europecv}


\end{document} 